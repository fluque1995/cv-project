\documentclass[11pt]{article}
\renewcommand{\baselinestretch}{1.05}
\usepackage[spanish]{babel}
\usepackage[utf8]{inputenc}
\usepackage{lipsum}

\usepackage{amsmath,amsthm,verbatim,amssymb,amsfonts,amscd}
\usepackage{graphicx, wrapfig}
\usepackage{float}
\usepackage{caption, subcaption}
\usepackage{tkz-fct}
\usetikzlibrary{babel}
\usepackage{pgfplots}
\usepackage{enumitem}
\usepackage{multicol, vwcol}
\usepackage{listingsutf8}
\usepackage{color}
\usepackage{hyperref}
\usepackage{booktabs}


\usepackage[sorting=none]{biblatex}
\bibliography{bibliography.bib}

\hypersetup{
    bookmarks=true,         % show bookmarks bar?
    unicode=false,          % non-Latin characters in Acrobat’s bookmarks
    pdftoolbar=true,        % show Acrobat’s toolbar?
    pdfmenubar=true,        % show Acrobat’s menu?
    pdffitwindow=false,     % window fit to page when opened
    pdfstartview={FitH},    % fits the width of the page to the window
    pdftitle={Inteligencia de Negocio - Práctica 3},    % title
    pdfauthor={Francisco Luque},     % author
    pdfsubject={Inteligencia de Negocio},   % subject of the document
    pdfnewwindow=true,      % links in new PDF window
    colorlinks=true,        % false: boxed links; true: colored links
    linkcolor=gray,          % color of internal links (change box color with linkbordercolor)
    citecolor=cyan,         % color of links to bibliography
    filecolor=magenta,      % color of file links
    urlcolor=blue           % color of external links
}

\setlength{\parindent}{0pt}
\topmargin0.0cm
\headheight0.0cm
\headsep0.0cm
\oddsidemargin0.0cm
\textheight23.0cm
\textwidth16.5cm
\footskip1.0cm
\theoremstyle{plain}

\newtheorem{theorem}{Teorema}
\newtheorem{corollary}{Corolario}
\newtheorem{lemma}{Lema}
\newtheorem{proposition}{Proposición}
\theoremstyle{definition}
\newtheorem{definition}{Definición}
\newtheorem{example}{Ejemplo}

\newcommand{\N}{\mathbb{N}}
\newcommand{\Z}{\mathbb{Z}}
\newcommand{\Q}{\mathbb{Q}}
\newcommand{\C}{\mathbb{C}}
\newcommand{\R}{\mathbb{R}}

\begin{document}

\title{Inteligencia de Negocio - Práctica 3\\
  DGIIM \\
  \large Problema de regresión - Competición en Kaggle}
\author{Francisco Luque Sánchez \\
  Universidad de Granada \\
  fluque1995@correo.ugr.es}

\begin{titlepage}
  \centering
  {\scshape\LARGE Universidad de Granada \par}
  \vspace{1cm}
  {\scshape\Large Visión por computación \par}
  \vspace{1.5cm}
  {\huge\bfseries Clasificación de imágenes usando CNNs \par}
  \vspace{2cm}
  {\Large\itshape Francisco Luque Sánchez \\
    María del Mar Ruiz Martín\par}
  \vspace{2cm}
  \includegraphics[width=.3\textwidth]{logougr.png}\par\vspace{1cm}
  % Bottom of the page
  {\large \today\par}
\end{titlepage}

\newpage

\tableofcontents

\newpage

\section{Introducción}

En esta práctica se tratará el problema de la clasificación de objetos
en imágenes, utilizando concretamente redes neuronales convolucionales
(\textit{CNNs}). El problema que se abordará consiste en tratar de
distinguir perros de gatos utilizando estos modelos. Se comenzará con
un modelo simple, el cual se irá modificando para tratar de mejorar su
capacidad para clasificar.

\subsection{Conjunto de datos utilizado}

El conjunto de datos utilizado se ha generado utilizando las bases de
datos mostradas en \cite{db1, db2, db3}. Se han extraído todas las
imágenes de las mismas y etiquetado en dos clases (perros y gatos),
obteniéndose un conjunto total de unos 13000 gatos y 25000
perros. Dicho conjunto se ha dividido en dos subconjuntos, un conjunto
de entrenamiento (unos 25500 ejemplos) y uno de test (en torno a 12500
ejemplos), tratando de mantener la proporción de perros y gatos lo más
parecida posible en ambos conjuntos.

\subsection{Aspectos de implementación}

Todo el código se ha desarrollado utilizando el \textit{framework}
TensorFlow \cite{tf}, que es una librería de código abierto
desarrollada por Google, orientada a la implementación de soluciones
utilizando inteligencia artificial. El primer modelo desarrollado, en
particular, se ha hecho utilizando un tutorial de la documentación
del \textit{framework}, que se puede consultar en \cite{cnn-tutorial}.

\printbibliography

\end{document}